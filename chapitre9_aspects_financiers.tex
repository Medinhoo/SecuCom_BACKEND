\chapter{Aspects financiers}

Cette section présente une estimation simplifiée des coûts et bénéfices de SecuCom dans le cadre de ce TFE.

\section{Coûts de développement}

\subsection{Coûts des ressources humaines}

Dans le cadre d'un projet étudiant, les coûts ont été estimés sur la base de tarifs appliqués à des profils juniors :
\begin{table}[h]
\centering
\begin{tabular}{|l|c|c|c|}
\hline
\textbf{Rôle} & \textbf{Tarif journalier} & \textbf{Jours} & \textbf{Coût total} \\
\hline
Développeur backend & 400€ & 20 & 8.000€ \\
Développeur frontend & 400€ & 15 & 6.000€ \\
Analyste & 400€ & 5 & 2.000€ \\
Tests & 400€ & 5 & 2.000€ \\
\hline
\textbf{Total} & & \textbf{45} & \textbf{18.000€} \\
\hline
\end{tabular}
\caption{Coûts des ressources humaines}
\end{table}

\subsection{Coûts d'infrastructure}

\begin{table}[h]
\centering
\begin{tabular}{|l|c|}
\hline
\textbf{Élément} & \textbf{Coût annuel} \\
\hline
Hébergement cloud & 360€ \\
Base de données & 240€ \\
Nom de domaine et SSL & 50€ \\
\hline
\textbf{Total} & \textbf{650€} \\
\hline
\end{tabular}
\caption{Coûts d'infrastructure}
\end{table}

\subsection{Récapitulatif}

\begin{table}[h]
\centering
\begin{tabular}{|l|c|}
\hline
\textbf{Catégorie} & \textbf{Coût} \\
\hline
Développement initial & 18.000€ \\
Infrastructure (première année) & 650€ \\
\hline
\textbf{Total investissement initial} & \textbf{18.650€} \\
\hline
\end{tabular}
\caption{Récapitulatif des coûts}
\end{table}

\section{Bénéfices attendus}

\subsection{Gains de productivité}

\begin{table}[h]
\centering
\begin{tabular}{|l|c|c|c|}
\hline
\textbf{Processus} & \textbf{Temps avant} & \textbf{Temps après} & \textbf{Gain} \\
\hline
Création d'entreprise & 45 min & 15 min & 67\% \\
Ajout de collaborateur & 30 min & 10 min & 67\% \\
Déclaration DIMONA & 20 min & 5 min & 75\% \\
Recherche d'informations & 15 min & 2 min & 87\% \\
\hline
\end{tabular}
\caption{Gains de productivité estimés}
\end{table}

Pour Sodabel (50 entreprises, 5 collaborateurs/entreprise, 200 DIMONA/an), ces gains représentent environ 500 heures économisées par an.

\subsection{Réduction des erreurs}

\begin{table}[h]
\centering
\begin{tabular}{|l|c|c|}
\hline
\textbf{Type d'erreur} & \textbf{Taux actuel} & \textbf{Taux attendu} \\
\hline
Erreurs de saisie & 5\% & < 1\% \\
DIMONA rejetées & 8\% & < 2\% \\
Informations manquantes & 12\% & < 3\% \\
\hline
\end{tabular}
\caption{Réduction des erreurs attendue}
\end{table}

\subsection{Bénéfices qualitatifs}

\begin{itemize}[leftmargin=*,label=\textcolor{darkgray}{$\bullet$},itemsep=0.3em]
  \item Image professionnelle améliorée
  \item Satisfaction client accrue
  \item Meilleure traçabilité des opérations
  \item Réduction du stress pour les employés
  \item Capacité à gérer plus de clients sans augmenter les ressources
\end{itemize}
