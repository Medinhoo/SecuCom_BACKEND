\chapter*{Glossaire}
\addcontentsline{toc}{chapter}{Glossaire}

\begin{description}[leftmargin=2cm, style=nextline]

\item[API (Application Programming Interface)] Interface de programmation qui permet à différentes applications de communiquer entre elles. Dans SecuCom, une API RESTful est utilisée pour la communication entre le frontend et le backend.

\item[BCE (Banque-Carrefour des Entreprises)] Registre contenant toutes les données d'identification des entreprises en Belgique. Chaque entreprise y reçoit un numéro d'identification unique.

\item[DIMONA (Déclaration Immédiate/Onmiddellijke Aangifte)] Système belge de déclaration obligatoire de tout engagement ou fin de relation de travail auprès de l'Office National de Sécurité Sociale (ONSS).

\item[DTO (Data Transfer Object)] Objet utilisé pour transporter des données entre différentes couches d'une application. Dans SecuCom, les DTOs sont utilisés pour la communication entre le frontend et le backend.

\item[Hibernate] Framework de mapping objet-relationnel (ORM) pour Java qui facilite la persistance des objets Java dans une base de données relationnelle.

\item[IBAN (International Bank Account Number)] Format international de numéro de compte bancaire utilisé pour identifier un compte bancaire de manière unique.

\item[JPA (Java Persistence API)] Spécification Java qui décrit la gestion des données relationnelles dans les applications Java. Dans SecuCom, Spring Data JPA est utilisé pour l'accès aux données.

\item[JWT (JSON Web Token)] Standard ouvert qui définit un format compact et autonome pour la transmission sécurisée d'informations entre parties sous forme d'objet JSON. Dans SecuCom, JWT est utilisé pour l'authentification et l'autorisation.

\item[Microservices] Style d'architecture qui structure une application comme un ensemble de services faiblement couplés. Bien que SecuCom ne soit pas actuellement basé sur une architecture microservices, cette approche pourrait être envisagée pour des évolutions futures.

\item[ONSS (Office National de Sécurité Sociale)] Organisme public belge chargé de la perception et de la gestion des cotisations sociales des employeurs et des travailleurs.

\item[ORM (Object-Relational Mapping)] Technique de programmation qui convertit les données entre des systèmes de types incompatibles dans des bases de données relationnelles et des langages de programmation orientés objet. Hibernate est l'ORM utilisé dans SecuCom.

\item[ReactJS] Bibliothèque JavaScript open-source utilisée pour construire des interfaces utilisateur, particulièrement pour les applications à page unique. ReactJS est utilisé pour le développement du frontend de SecuCom.

\item[RGPD (Règlement Général sur la Protection des Données)] Règlement de l'Union européenne qui constitue le texte de référence en matière de protection des données à caractère personnel. SecuCom est conçu pour être conforme au RGPD.

\item[Secrétariat social] En Belgique, organisme privé agréé qui aide les employeurs à accomplir leurs obligations administratives et sociales liées à l'emploi de personnel.

\item[Spring Boot] Framework Java qui simplifie le développement d'applications Java en fournissant une configuration par défaut pour les projets Spring. Spring Boot est utilisé pour le développement du backend de SecuCom.

\item[Spring Security] Module du framework Spring qui fournit des services d'authentification et d'autorisation pour les applications Java. Spring Security est utilisé dans SecuCom pour la gestion de la sécurité.

\item[SRL (Société à Responsabilité Limitée)] Forme juridique d'entreprise en Belgique où la responsabilité des associés est limitée à leurs apports. Sodabel est une SRL.

\item[TypeScript] Langage de programmation open-source développé par Microsoft qui ajoute le typage statique optionnel à JavaScript. TypeScript est utilisé dans le frontend de SecuCom pour améliorer la maintenabilité et la détection d'erreurs.

\item[UML (Unified Modeling Language)] Langage de modélisation graphique utilisé en génie logiciel pour visualiser, spécifier, construire et documenter les artefacts d'un système. Plusieurs diagrammes UML sont utilisés dans ce document pour illustrer l'architecture de SecuCom.

\end{description}
