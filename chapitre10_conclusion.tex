\chapter*{Conclusion}
\addcontentsline{toc}{chapter}{Conclusion}
\markboth{Conclusion}{}

Cette section présente une synthèse du projet SecuCom, en abordant les difficultés rencontrées, le bilan global et les perspectives d'évolution future.

\section*{Défis rencontrées}
\addcontentsline{toc}{section}{Difficultés rencontrées}
\markright{Difficultés rencontrées}

Le développement de SecuCom a présenté plusieurs défis techniques et conceptuels qui ont nécessité des solutions innovantes :

\begin{itemize}[leftmargin=*,label=\textcolor{darkgray}{$\bullet$},itemsep=0.3em]
  \item \textbf{Complexité du domaine métier} : La compréhension approfondie des processus d'un secrétariat social belge, notamment les spécificités des déclarations DIMONA et les exigences légales associées, a représenté un défi initial important. Cette complexité a nécessité de nombreux échanges avec Sodabel pour s'assurer que l'application réponde précisément aux besoins réels.

  \item \textbf{Compréhension des besoins du client} : L'un des défis majeurs a été de bien cerner les attentes et besoins réels de Sodabel, au-delà des demandes initiales parfois trop ambitieuses. Ce processus a nécessité une communication constante et une capacité d'analyse pour distinguer les besoins essentiels des fonctionnalités secondaires.

  \item \textbf{Savoir dire non et gérer le périmètre} : Apprendre à refuser certaines demandes lorsqu'elles dépassaient le cadre réalisable du projet a constitué une difficulté importante. Cette compétence s'est avérée cruciale pour maintenir le projet dans des limites réalistes en termes de temps et de ressources disponibles.

  \item \textbf{Sécurisation des données sensibles} : La manipulation de données personnelles et professionnelles confidentielles a imposé la mise en place d'une architecture de sécurité robuste. L'implémentation du système d'authentification JWT et la gestion fine des autorisations basées sur les rôles ont demandé une attention particulière pour garantir la confidentialité et l'intégrité des données.

  \item \textbf{Séparation des espaces utilisateurs} : La création d'espaces distincts pour le secrétariat social et les entreprises clientes, tout en maintenant une cohérence dans l'expérience utilisateur, a constitué un défi architectural. Il a fallu concevoir un système où les données sont strictement cloisonnées tout en permettant une navigation fluide.

  \item \textbf{Équilibre entre ambition et faisabilité} : Le projet initial comportait un nombre important de fonctionnalités qui ont dû être réduites pour s'adapter aux contraintes de temps et de ressources. Cette situation a imposé un exercice constant de priorisation et de recentrage sur l'essentiel, tout en concevant une architecture suffisamment modulaire pour permettre des extensions futures.
\end{itemize}

\section*{Bilan}
\addcontentsline{toc}{section}{Bilan}
\markright{Bilan}

\textbf{Le développement de SecuCom peut être considéré comme un succès dans la mesure où les objectifs initiaux ont été atteints :}

\begin{itemize}[leftmargin=*,label=\textcolor{darkgray}{$\bullet$},itemsep=0.3em]
  \item \textbf{Réponse aux besoins identifiés} : L'application répond directement aux problématiques concrètes identifiées chez Sodabel, notamment la digitalisation des processus manuels, la centralisation des communications et la réduction des risques d'erreurs. Les gains de productivité estimés (environ 67\% pour la création d'entreprise, environ 75\% pour les déclarations DIMONA selon les estimations) démontrent la pertinence de la solution.

  \item \textbf{Architecture technique solide} : L'utilisation de technologies modernes et éprouvées (Spring Boot, ReactJS, JWT) a permis de construire une base technique robuste, sécurisée et évolutive. La séparation claire des responsabilités entre les différentes couches de l'application facilite sa maintenance et son extension future.

  \item \textbf{Approche modulaire réussie} : Malgré la réduction du périmètre initial, l'architecture modulaire adoptée permet d'envisager sereinement l'ajout de nouvelles fonctionnalités sans remettre en question les fondations du système. Cette approche s'est avérée être un choix judicieux face aux contraintes du projet.

  \item \textbf{Positionnement stratégique} : Face aux solutions existantes comme EasyPay et Liantis, SecuCom se distingue par son approche ciblée et minimaliste, répondant spécifiquement aux besoins des petits secrétariats sociaux. Cette spécialisation constitue un avantage concurrentiel dans un marché dominé par des solutions généralistes souvent surdimensionnées.

  \item \textbf{Apprentissage de la priorisation} : Le processus de développement a permis d'acquérir une compétence essentielle : la capacité à identifier et prioriser les fonctionnalités vraiment essentielles, plutôt que de se disperser dans une multitude de fonctionnalités secondaires. Cette leçon constitue un acquis précieux pour les projets futurs.
\end{itemize}

Ce projet a également permis de démontrer qu'une approche ciblée, privilégiant la qualité et la pertinence des fonctionnalités plutôt que leur quantité, peut apporter une valeur significative dans un domaine aussi complexe que celui des secrétariats sociaux.

\section*{Améliorations futures}
\addcontentsline{toc}{section}{Améliorations futures}
\markright{Améliorations futures}

SecuCom a été conçu dès le départ comme une plateforme évolutive, destinée à s'enrichir progressivement de nouvelles fonctionnalités. Plusieurs axes d'amélioration ont été identifiés pour les développements futurs, notamment ceux qui avaient été envisagés initialement mais qui ont dû être reportés pour respecter les contraintes du projet :

\newpage
\subsection*{Demandes de services}
\addcontentsline{toc}{subsection}{Demandes de services}

L'implémentation d'un système de demandes de services permettrait aux entreprises clientes de solliciter directement via la plateforme différents types de prestations auprès du secrétariat social :

\begin{itemize}[leftmargin=*,label=\textcolor{darkgray}{$\bullet$},itemsep=0.3em]
  \item Module de création de demandes avec catégorisation (conseil juridique, modification administrative, attestation spécifique, etc.)
  \item Système de suivi en temps réel de l'état d'avancement des demandes
  \item Notifications automatiques informant les clients des mises à jour
  \item Historique des demandes permettant de consulter les échanges passés
  \item Tableau de bord pour le secrétariat social facilitant la gestion et la priorisation des demandes
\end{itemize}

Cette fonctionnalité remplacerait avantageusement les échanges actuels par email et WhatsApp, centralisant toutes les communications dans un espace structuré et traçable.

\subsection*{Gestion et génération de documents}
\addcontentsline{toc}{subsection}{Gestion et génération de documents}

Le développement d'un système complet de gestion documentaire constituerait une amélioration majeure :

\begin{itemize}[leftmargin=*,label=\textcolor{darkgray}{$\bullet$},itemsep=0.3em]
  \item Génération automatique de documents officiels (contrats de travail, C4, fiches de paie) à partir des données du système
  \item Modèles de documents personnalisables adaptés aux besoins spécifiques de chaque entreprise
  \item Système de signature électronique pour faciliter la validation des documents
  \item Archivage sécurisé avec classification et recherche avancée
  \item Gestion des versions permettant de suivre l'évolution des documents
\end{itemize}

Cette fonctionnalité réduirait considérablement le temps consacré à la création manuelle de documents et minimiserait les risques d'erreurs dans leur production.

\subsection*{Facturation automatique}
\addcontentsline{toc}{subsection}{Facturation automatique}

L'intégration d'un système de facturation automatique pour les services et documents fournis via la plateforme apporterait une valeur ajoutée significative :

\begin{itemize}[leftmargin=*,label=\textcolor{darkgray}{$\bullet$},itemsep=0.3em]
  \item Génération automatique de factures basée sur les services rendus et les documents produits
  \item Paramétrage flexible des tarifs selon le type de client et de service
  \item Suivi des paiements avec relances automatiques
  \item Tableau de bord financier offrant une vue d'ensemble des revenus
  \item Intégration avec des solutions comptables pour faciliter la réconciliation
\end{itemize}

Cette fonctionnalité permettrait non seulement d'optimiser le processus de facturation, mais aussi d'améliorer le suivi financier des prestations du secrétariat social.

\subsection*{Autres améliorations potentielles}
\addcontentsline{toc}{subsection}{Autres améliorations potentielles}

Au-delà des trois axes principaux mentionnés, d'autres améliorations pourraient enrichir SecuCom :

\begin{itemize}[leftmargin=*,label=\textcolor{darkgray}{$\bullet$},itemsep=0.3em]
  \item Intégration directe avec l'ONSS pour automatiser complètement les déclarations DIMONA
  \item Tableau de bord analytique offrant des insights sur l'activité des entreprises clientes
  \item Système de chat intégré pour les communications en temps réel entre le secrétariat et ses clients
  \item Implémentation du protocole HTTPS pour sécuriser les communications entre le client et le serveur, essentielle dans l'objectif d'un déploiement en production serein et conforme aux standards de sécurité actuels
\end{itemize}

\section*{Conclusion générale}
\addcontentsline{toc}{section}{Conclusion générale}
\markright{Conclusion générale}

SecuCom représente bien plus qu'un simple projet académique ; il constitue une réponse concrète et viable à des problématiques réelles rencontrées par les secrétariats sociaux de petite taille. En privilégiant une approche ciblée et minimaliste, le projet a su apporter une valeur ajoutée immédiate tout en posant les fondations d'une évolution future.\\

Les fonctionnalités actuellement implémentées (gestion des entreprises, des collaborateurs et des déclarations DIMONA) répondent aux besoins les plus critiques identifiés lors de l'analyse initiale. Les améliorations futures envisagées, notamment les demandes de services, la gestion documentaire et la facturation automatique, s'inscrivent dans une vision cohérente d'évolution progressive de la plateforme.\\

L'une des leçons les plus importantes tirées de ce projet est l'importance de la modularité et de la priorisation. Face à des ambitions initiales trop vastes, la capacité à recentrer le projet sur les fonctionnalités essentielles tout en concevant une architecture permettant des extensions futures s'est avérée déterminante pour atteindre les objectifs fixés.\\

En définitive, SecuCom illustre comment une solution informatique bien conçue, même avec un périmètre fonctionnel volontairement limité, peut transformer des processus métier traditionnels, apportant efficacité, sécurité et satisfaction tant aux prestataires de services qu'à leurs clients. La véritable réussite du projet réside peut-être moins dans l'étendue des fonctionnalités développées que dans la solidité des fondations posées pour l'avenir.\\
