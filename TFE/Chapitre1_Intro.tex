\chapter*{Introduction}
\addcontentsline{toc}{chapter}{Introduction}
\markboth{Introduction}{}

À l'ère numérique, la digitalisation des processus administratifs est devenue essentielle pour les entreprises. Les secrétariats sociaux belges font face à des défis importants liés aux réglementations complexes et au traitement des déclarations DIMONA. Malgré cela, de nombreux secrétariats sociaux indépendants utilisent encore des processus manuels, source d'inefficacités et d'erreurs.

\section*{Objectifs du TFE}
\addcontentsline{toc}{section}{Objectifs du TFE}
\markright{Objectifs du TFE}

Ce Travail de Fin d'Études, présenté à l'ISFCE dans le cadre de l'obtention du diplôme de Bachelier en informatique à orientation développement d'applications, vise à concevoir, développer et valider une plateforme de gestion pour secrétariats sociaux, baptisée \textbf{SecuCom}. Cette solution est spécifiquement adaptée aux besoins de Sodabel, un secrétariat social indépendant avec lequel une collaboration étroite a été établie tout au long du projet.

\vspace{0.5cm}

\begin{tcolorbox}[
  title={\textbf{SecuCom}},
  colback=blue!5!white,
  colframe=primarycolor,
  fonttitle=\bfseries,
  boxrule=0.5mm,
  arc=2mm,
  left=6mm,
  right=6mm,
  top=6mm,
  bottom=6mm
]\textbf{SecuCom} est une plateforme web sécurisée conçue pour simplifier et optimiser les processus administratifs des secrétariats sociaux indépendants. Elle se concentre sur les fonctionnalités essentielles, évitant la complexité souvent rencontrée dans les solutions existantes sur le marché.
\end{tcolorbox}

\vspace{0.5cm}


\noindent Plus précisément, les objectifs de ce TFE sont :
\begin{itemize}[leftmargin=*,label=\textcolor{darkgray}{$\bullet$},itemsep=0.3em]
  \item D'analyser les processus internes de Sodabel concernant la gestion des entreprises clientes, de leurs employés et des déclarations DIMONA
  \item De concevoir une architecture backend robuste et sécurisée répondant aux besoins spécifiques identifiés
  \item D'implémenter les fonctionnalités clés permettant de fluidifier les processus d'encodage et de gestion
  \item De valider la solution à travers des tests fonctionnels et de sécurité
  \item De fournir un premier jet d'une solution qui pourra être développée davantage dans un cadre professionnel futur
\end{itemize}

\noindent L'ambition de ce projet dépasse le simple cadre académique. Il s'agit de proposer une solution réellement opérationnelle qui pourra être déployée et améliorée progressivement.


\section*{Méthodologie}
\addcontentsline{toc}{section}{Méthodologie}
\markright{Méthodologie}

\noindent Pour atteindre ces objectifs, ce travail s'articule autour de quatre étapes principales, organisées selon une approche méthodique et rigoureuse.

\begin{enumerate}[leftmargin=*,itemsep=0.3em]
  \item \textbf{Analyse des besoins} : Entretiens avec les responsables de Sodabel pour identifier les points de friction dans les processus actuels (WhatsApp, email) et leurs limitations.

  \item \textbf{Conception} : Élaboration des spécifications fonctionnelles et techniques avec modélisation UML, en tenant compte des contraintes du projet.

  \item \textbf{Implémentation} : Développement backend suivant les bonnes pratiques, avec focus sur la sécurité des données et la séparation des espaces clients.

  \item \textbf{Validation} : Tests des fonctionnalités pour garantir que la solution répond aux problématiques identifiées.
\end{enumerate}

\begin{note}
Une approche itérative a été privilégiée tout au long du projet, permettant des retours réguliers vers les phases précédentes pour affiner la solution en fonction des retours utilisateurs et des contraintes techniques identifiées en cours de développement. Cette flexibilité méthodologique a permis d'adapter la solution aux besoins réels du terrain.
\end{note}

\section*{Structure du document}
\addcontentsline{toc}{section}{Structure du document}
\markright{Structure du document}

Ce TFE suit une progression logique depuis l'analyse initiale jusqu'à l'évaluation finale, structurée comme suit :

\begin{itemize}[leftmargin=*,label=\textcolor{darkgray}{$\bullet$},itemsep=0.3em]
  \item \textbf{Contexte} : Présentation de l'environnement des secrétariats sociaux en Belgique et des problématiques spécifiques de Sodabel.
  
  \item \textbf{Description du sujet} : Exposition de SecuCom, ses objectifs et son fonctionnement, avec accent sur sa simplicité d'utilisation.
  
  \item \textbf{Analyse de l'existant} : Comparaison avec les solutions du marché et positionnement de SecuCom comme alternative ciblée.
  
  \item \textbf{Exigences, besoins et Analyse} : Identification des besoins et leur traduction en architecture système via des diagrammes UML.
  
  \item \textbf{Conception et Développement} : Présentation des choix architecturaux et de l'implémentation des fonctionnalités principales.
  
  \item \textbf{Aspects financiers et Conclusion} : Évaluation économique du projet et perspectives d'évolution future.
\end{itemize}
